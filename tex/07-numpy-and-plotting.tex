\documentclass[english,serif,mathserif,xcolor=pdftex,dvipsnames,table]{beamer}

\usepackage[T1]{fontenc}
\usepackage[utf8]{inputenc}

\usetheme[informal]{s3it}
\usepackage{s3it}

\title[3. Sequences and loops]{%
  A Short and Incomplete Introduction to Python
}
\subtitle{\bfseries Part 3: Sequences and \texttt{for}-loops}
\author[R.~Murri]{%
  Riccardo Murri \texttt{<riccardo.murri@uzh.ch>}
  \\
  S3IT: Services and Support for Science IT,
  \\
  University of Zurich
}
\date{June~8, 2017}


\begin{document}

% title frame
\maketitle


\part{NumPy}

NumPy is a Python package that provides:

\begin{itemize}
\tightlist
\item
  a multi-dimensional \emph{array} and whole-array operations
\item
  fast matrix and vector operations
\item
  a library of mathematical functions
\end{itemize}

To use NumPy in your code, you first need to import it:

\begin{Shaded}
\begin{Highlighting}[]
\ImportTok{import} \NormalTok{numpy }\ImportTok{as} \NormalTok{np}
\end{Highlighting}
\end{Shaded}

After this \texttt{import\ ...\ as} statement, you can use all NumPy
functions by prefixing them with \texttt{np.}

\subsection{NumPy arrays}\label{numpy-arrays}

NumPy arrays (also called \emph{ndarray}) are
\href{https://docs.scipy.org/doc/numpy/user/basics.creation.html}{constructed}
with function \texttt{np.array}::

\begin{Shaded}
\begin{Highlighting}[]
\NormalTok{a }\OperatorTok{=} \NormalTok{np.array([}\DecValTok{1}\NormalTok{, }\DecValTok{2}\NormalTok{, }\DecValTok{3}\NormalTok{, }\DecValTok{4}\NormalTok{, }\DecValTok{5}\NormalTok{])}
\end{Highlighting}
\end{Shaded}

They behave behave a lot like Python lists:

\begin{Shaded}
\begin{Highlighting}[]
\CommentTok{# get items using the [] operator}
\NormalTok{a[}\DecValTok{0}\NormalTok{]}
\end{Highlighting}
\end{Shaded}

\begin{verbatim}
1
\end{verbatim}

\begin{Shaded}
\begin{Highlighting}[]
\CommentTok{# get a slice using [:]}
\NormalTok{a[}\DecValTok{0}\NormalTok{:}\DecValTok{3}\NormalTok{]}
\end{Highlighting}
\end{Shaded}

\begin{verbatim}
array([1, 2, 3])
\end{verbatim}

Note that again, a slice of a \emph{ndarray} is again a \emph{ndarray}.

\begin{Shaded}
\begin{Highlighting}[]
\CommentTok{# set an item using [] =}
\NormalTok{a[}\DecValTok{4}\NormalTok{] }\OperatorTok{=} \DecValTok{0}
\end{Highlighting}
\end{Shaded}

\begin{Shaded}
\begin{Highlighting}[]
\BuiltInTok{print}\NormalTok{(a)}
\end{Highlighting}
\end{Shaded}

\begin{verbatim}
[1 2 3 4 0]
\end{verbatim}

Note that \emph{ndarray}'s are \textbf{homogeneous} -- you cannot mix
e.g.~integers and strings, but neither integers and floats:

\begin{Shaded}
\begin{Highlighting}[]
\NormalTok{a[}\DecValTok{4}\NormalTok{] }\OperatorTok{=} \StringTok{'five'}
\end{Highlighting}
\end{Shaded}

\begin{verbatim}
ValueErrorTraceback (most recent call last)

<ipython-input-7-6558a4aae2db> in <module>()
----> 1 a[4] = 'five'


ValueError: invalid literal for long() with base 10: 'five'
\end{verbatim}

\begin{Shaded}
\begin{Highlighting}[]
\CommentTok{# here, the `5.5` is automatically converted to integer}
\NormalTok{a[}\DecValTok{4}\NormalTok{] }\OperatorTok{=} \FloatTok{5.5}
\end{Highlighting}
\end{Shaded}

\begin{Shaded}
\begin{Highlighting}[]
\NormalTok{a}
\end{Highlighting}
\end{Shaded}

\begin{verbatim}
array([1, 2, 3, 4, 5])
\end{verbatim}

\subsubsection{Setting the type of array
items}\label{setting-the-type-of-array-items}

Arrays are homogeneous (all elements must have the same type) and the
\href{https://docs.scipy.org/doc/numpy/user/basics.types.html}{type} is
set at array creation time:

\begin{Shaded}
\begin{Highlighting}[]
\NormalTok{a }\OperatorTok{=} \NormalTok{np.array([}\DecValTok{1}\NormalTok{, }\DecValTok{2}\NormalTok{, }\DecValTok{3}\NormalTok{, }\DecValTok{4}\NormalTok{, }\DecValTok{5}\NormalTok{], dtype}\OperatorTok{=}\NormalTok{np.float64)}

\NormalTok{a[}\DecValTok{4}\NormalTok{] }\OperatorTok{=} \FloatTok{5.5}

\CommentTok{# show a}
\NormalTok{a}
\end{Highlighting}
\end{Shaded}

\begin{verbatim}
array([ 1. ,  2. ,  3. ,  4. ,  5.5])
\end{verbatim}

\subsection{Plotting}\label{plotting}

\href{http://matplotlib.org/gallery.html}{Matplotlib} is the most-used
plotting library in the Python community: it provides a large array of
(mostly low level) facilities for making plots, and a more high-level
interface largely inspired by MATLAB plotting system.

\href{http://seaborn.pydata.org/index.html}{Seaborn} is an add-on
library that provides:

\begin{itemize}
\tightlist
\item
  better default visual styles
\item
  easier plotting functions for many commonly-used types of plots
\end{itemize}

\begin{Shaded}
\begin{Highlighting}[]
\OperatorTok{%}\NormalTok{matplotlib inline}

\ImportTok{import} \NormalTok{matplotlib.pyplot }\ImportTok{as} \NormalTok{plt}
\ImportTok{import} \NormalTok{seaborn}
\end{Highlighting}
\end{Shaded}

The \texttt{np.arange()} function can be used to generate an
\emph{ndarray} containing equally-spaced points from an interval on the
real line:

\begin{Shaded}
\begin{Highlighting}[]
\CommentTok{# points in the interval [0,4) spaces 0.1 apart }
\NormalTok{x }\OperatorTok{=} \NormalTok{np.arange(}\DecValTok{0}\NormalTok{, }\DecValTok{4}\NormalTok{, }\FloatTok{0.1}\NormalTok{)}
\end{Highlighting}
\end{Shaded}

\begin{Shaded}
\begin{Highlighting}[]
\NormalTok{x}
\end{Highlighting}
\end{Shaded}

\begin{verbatim}
array([ 0. ,  0.1,  0.2,  0.3,  0.4,  0.5,  0.6,  0.7,  0.8,  0.9,  1. ,
        1.1,  1.2,  1.3,  1.4,  1.5,  1.6,  1.7,  1.8,  1.9,  2. ,  2.1,
        2.2,  2.3,  2.4,  2.5,  2.6,  2.7,  2.8,  2.9,  3. ,  3.1,  3.2,
        3.3,  3.4,  3.5,  3.6,  3.7,  3.8,  3.9])
\end{verbatim}

We can easily create an array with constant value:

\begin{Shaded}
\begin{Highlighting}[]
\NormalTok{y1 }\OperatorTok{=} \NormalTok{np.ones(}\BuiltInTok{len}\NormalTok{(x))}
\end{Highlighting}
\end{Shaded}

\begin{Shaded}
\begin{Highlighting}[]
\NormalTok{y1}
\end{Highlighting}
\end{Shaded}

\begin{verbatim}
array([ 1.,  1.,  1.,  1.,  1.,  1.,  1.,  1.,  1.,  1.,  1.,  1.,  1.,
        1.,  1.,  1.,  1.,  1.,  1.,  1.,  1.,  1.,  1.,  1.,  1.,  1.,
        1.,  1.,  1.,  1.,  1.,  1.,  1.,  1.,  1.,  1.,  1.,  1.,  1.,  1.])
\end{verbatim}

A line plot of y1 over x will just show a flat horizontal line:

\begin{Shaded}
\begin{Highlighting}[]
\NormalTok{plt.plot(x, y1)}
\end{Highlighting}
\end{Shaded}

\begin{verbatim}
[<matplotlib.lines.Line2D at 0x7f2ef1dd8990>]
\end{verbatim}

\includegraphics{output_27_1.png}

The \texttt{np.linspace()} function works just like \texttt{np.arange()}
but its third parameter is the \emph{number of subdivisions of the
interval} (as opposed to the difference of two consecutive points):

\begin{Shaded}
\begin{Highlighting}[]
\NormalTok{y2 }\OperatorTok{=} \NormalTok{np.linspace(}\FloatTok{0.5}\NormalTok{, }\DecValTok{3}\NormalTok{, }\BuiltInTok{len}\NormalTok{(x))}
\end{Highlighting}
\end{Shaded}

\begin{Shaded}
\begin{Highlighting}[]
\NormalTok{y2}
\end{Highlighting}
\end{Shaded}

\begin{verbatim}
array([ 0.5       ,  0.56410256,  0.62820513,  0.69230769,  0.75641026,
        0.82051282,  0.88461538,  0.94871795,  1.01282051,  1.07692308,
        1.14102564,  1.20512821,  1.26923077,  1.33333333,  1.3974359 ,
        1.46153846,  1.52564103,  1.58974359,  1.65384615,  1.71794872,
        1.78205128,  1.84615385,  1.91025641,  1.97435897,  2.03846154,
        2.1025641 ,  2.16666667,  2.23076923,  2.29487179,  2.35897436,
        2.42307692,  2.48717949,  2.55128205,  2.61538462,  2.67948718,
        2.74358974,  2.80769231,  2.87179487,  2.93589744,  3.        ])
\end{verbatim}

\begin{Shaded}
\begin{Highlighting}[]
\NormalTok{plt.plot(x, y2)}
\end{Highlighting}
\end{Shaded}

\begin{verbatim}
[<matplotlib.lines.Line2D at 0x7f2eee196f50>]
\end{verbatim}

\includegraphics{output_31_1.png}

\subsubsection{Exercise 1.}\label{exercise-1.}

How can you modify the plotting code above to make the line red? How can
you make it thicker?

\begin{Shaded}
\begin{Highlighting}[]
\CommentTok{# hint: look at the docs for `matplotlib.pyplot.plot`}
\end{Highlighting}
\end{Shaded}

Plotting different series of data in the same figure requires a bit more
work:

\begin{enumerate}
\tightlist
\item
  First use the
  \href{http://matplotlib.org/api/pyplot_api.html\#matplotlib.pyplot.subplots}{\texttt{plt.subplots}}
  function to create \emph{figure} and an \emph{axes} object
\item
  An
  \href{http://matplotlib.org/api/axes_api.html\#matplotlib.axes.Axes}{\emph{axes}
  object} is a ``frame'' for a single plot -- use methods
  \href{http://matplotlib.org/api/_as_gen/matplotlib.axes.Axes.plot.html\#matplotlib.axes.Axes.plot}{\texttt{.plot()}}
  to lay a graph onto the canvas. Each invocation of
  \href{http://matplotlib.org/api/_as_gen/matplotlib.axes.Axes.plot.html\#matplotlib.axes.Axes.plot}{\texttt{.plot()}}
  \emph{adds} a plot onto the canvas.
\item
  The
  \href{http://matplotlib.org/api/figure_api.html\#matplotlib.figure.Figure}{\emph{figure}
  object} contains all the axes can be used for saving the final output
  with
  \href{http://matplotlib.org/api/figure_api.html\#matplotlib.figure.Figure.savefig}{\texttt{.savefig()}}
\end{enumerate}

\begin{Shaded}
\begin{Highlighting}[]
\NormalTok{fig, ax }\OperatorTok{=} \NormalTok{plt.subplots(}\DecValTok{1}\NormalTok{, }\DecValTok{1}\NormalTok{, figsize}\OperatorTok{=}\NormalTok{[}\DecValTok{10}\NormalTok{, }\DecValTok{7}\NormalTok{])}

\NormalTok{ax.plot(x, y1)}
\NormalTok{ax.plot(x, y2)}
\end{Highlighting}
\end{Shaded}

\begin{verbatim}
[<matplotlib.lines.Line2D at 0x7f2eedfd9610>]
\end{verbatim}

\includegraphics{output_35_1.png}

\subsection{Whole-array operations}\label{whole-array-operations}

NumPy arrays implement operators \texttt{+}, \texttt{-}, \texttt{*},
\texttt{/} as \emph{element-wise operations}.

For instance, the array sum \texttt{a+b} is the array whose
\texttt{i}-th element is \texttt{a{[}i{]}+b{[}i{]}}:

\begin{Shaded}
\begin{Highlighting}[]
\NormalTok{a }\OperatorTok{=} \NormalTok{np.array([}\DecValTok{1}\NormalTok{, }\DecValTok{2}\NormalTok{, }\DecValTok{3}\NormalTok{])}
\NormalTok{b }\OperatorTok{=} \NormalTok{np.array([}\FloatTok{0.1}\NormalTok{, }\FloatTok{0.2}\NormalTok{, }\FloatTok{0.3}\NormalTok{])}
\NormalTok{c }\OperatorTok{=} \NormalTok{a }\OperatorTok{+} \NormalTok{b}
\end{Highlighting}
\end{Shaded}

\begin{Shaded}
\begin{Highlighting}[]
\NormalTok{c}
\end{Highlighting}
\end{Shaded}

\begin{verbatim}
array([ 1.1,  2.2,  3.3])
\end{verbatim}

Operations involving an array and a scalar value promote the scalar to a
constant array having the same \emph{shape}:

\begin{Shaded}
\begin{Highlighting}[]
\NormalTok{y_a }\OperatorTok{=} \NormalTok{y2 }\OperatorTok{+} \FloatTok{1.0}
\NormalTok{y_b }\OperatorTok{=} \NormalTok{y2 }\OperatorTok{-} \FloatTok{1.0}
\end{Highlighting}
\end{Shaded}

\begin{Shaded}
\begin{Highlighting}[]
\NormalTok{fig, ax }\OperatorTok{=} \NormalTok{plt.subplots(}\DecValTok{1}\NormalTok{, }\DecValTok{1}\NormalTok{, figsize}\OperatorTok{=}\NormalTok{[}\DecValTok{10}\NormalTok{, }\DecValTok{7}\NormalTok{])}

\NormalTok{ax.plot(x, y_a, linestyle}\OperatorTok{=}\StringTok{'-.'}\NormalTok{)}
\NormalTok{ax.plot(x, y2)}
\NormalTok{ax.plot(x, y_b, linestyle}\OperatorTok{=}\StringTok{'-.'}\NormalTok{)}
\end{Highlighting}
\end{Shaded}

\begin{verbatim}
[<matplotlib.lines.Line2D at 0x7f2eedfafb90>]
\end{verbatim}

\includegraphics{output_42_1.png}

\subsubsection{Exercise 2.}\label{exercise-2.}

Plot the graph of function \$f(x) = 1/x\$

\begin{Shaded}
\begin{Highlighting}[]
\NormalTok{y }\OperatorTok{=} \FloatTok{1.0}\OperatorTok{/}\NormalTok{x}
\end{Highlighting}
\end{Shaded}

\subsection{Matrices and higher-dimensional
arrays}\label{matrices-and-higher-dimensional-arrays}

NumPy allows creation and manipulation of multi-dimensional arrays, the
prime example being 2D matrices.

The simplest way to initialize a matrix is to pass \texttt{np.array} a
list of the matrix rows:

\begin{Shaded}
\begin{Highlighting}[]
\NormalTok{m }\OperatorTok{=} \NormalTok{np.array([[}\DecValTok{0}\NormalTok{, }\DecValTok{1}\NormalTok{], [}\OperatorTok{-}\DecValTok{1}\NormalTok{, }\DecValTok{0}\NormalTok{]])}
\end{Highlighting}
\end{Shaded}

\begin{Shaded}
\begin{Highlighting}[]
\NormalTok{m}
\end{Highlighting}
\end{Shaded}

\begin{verbatim}
array([[ 0,  1],
       [-1,  0]])
\end{verbatim}

The \emph{shape} of a multi-dimensional array is the number of elements
along each dimension:

\begin{Shaded}
\begin{Highlighting}[]
\NormalTok{m.shape}
\end{Highlighting}
\end{Shaded}

\begin{verbatim}
(2, 2)
\end{verbatim}

For a 1-dimensional array, the shape is a list comprising a single
number:

\begin{Shaded}
\begin{Highlighting}[]
\NormalTok{x.shape}
\end{Highlighting}
\end{Shaded}

Differently from native Python lists, indexing of multi-dimensional
arrays is done with the syntax \texttt{m{[}i,j{]}} instead of
\texttt{m{[}i{]}{[}j{]}}:

\begin{Shaded}
\begin{Highlighting}[]
\NormalTok{m[}\DecValTok{1}\NormalTok{,}\DecValTok{1}\NormalTok{] }\OperatorTok{=} \DecValTok{7}
\end{Highlighting}
\end{Shaded}

\begin{Shaded}
\begin{Highlighting}[]
\NormalTok{m}
\end{Highlighting}
\end{Shaded}

You can take the \emph{transpose} of a matrix by using method
\texttt{.T} (note: has to be \emph{capital} T):

\begin{Shaded}
\begin{Highlighting}[]
\NormalTok{m.T}
\end{Highlighting}
\end{Shaded}

\begin{verbatim}
array([[ 0, -1],
       [ 1,  7]])
\end{verbatim}

Functions are available to create special matrices (see
\href{https://docs.scipy.org/doc/numpy/reference/generated/numpy.eye.html\#numpy.eye}{here}
for a complete reference):

\begin{itemize}
\tightlist
\item
  \href{https://docs.scipy.org/doc/numpy/reference/generated/numpy.diag.html\#numpy.diag}{\texttt{np.diag(a)}}
  return a square matrix having entries of array \texttt{a} along the
  diagonal. (Note that the length of \texttt{a} determines the dimension
  of the matrix.)
\item
  \href{https://docs.scipy.org/doc/numpy/reference/generated/numpy.eye.html\#numpy.eye}{\texttt{np.eye(N,M)}}
  return the ``identity'' matrix of dimension \$N \textbackslash{}times
  M\$
\end{itemize}

\subsubsection{Matrix-Vector and Matrix-Matrix
multiplication}\label{matrix-vector-and-matrix-matrix-multiplication}

The
\href{https://docs.scipy.org/doc/numpy/reference/generated/numpy.dot.html}{\texttt{np.dot}}
is used to compute:

\begin{enumerate}
\tightlist
\item
  the dot (inner) product of two vectors
\item
  the matrix-vector product
\item
  matrix-matrix product
\end{enumerate}

\begin{Shaded}
\begin{Highlighting}[]
\NormalTok{a }\OperatorTok{=} \NormalTok{np.array([}\DecValTok{0}\NormalTok{, }\DecValTok{1}\NormalTok{])}
\NormalTok{b }\OperatorTok{=} \NormalTok{np.array([}\DecValTok{1}\NormalTok{, }\DecValTok{0}\NormalTok{])}
\end{Highlighting}
\end{Shaded}

The dot product of vectors \texttt{a} and \texttt{b} is 0:

\begin{Shaded}
\begin{Highlighting}[]
\NormalTok{np.dot(a, b)}
\end{Highlighting}
\end{Shaded}

\begin{verbatim}
0
\end{verbatim}

\begin{Shaded}
\begin{Highlighting}[]
\NormalTok{np.dot(a, a)}
\end{Highlighting}
\end{Shaded}

\begin{verbatim}
1
\end{verbatim}

When \texttt{np.dot} is applied to a matrix and a vector, the usual
matrix-vector product results:

\begin{Shaded}
\begin{Highlighting}[]
\NormalTok{np.dot(m, a)}
\end{Highlighting}
\end{Shaded}

\begin{Shaded}
\begin{Highlighting}[]
\NormalTok{np.dot(m, b)}
\end{Highlighting}
\end{Shaded}

Finally, when \texttt{np.dot} is applied to two matrices, the regular
matrix-matrix multiplication takes place:

\begin{Shaded}
\begin{Highlighting}[]
\NormalTok{np.dot(m, m)}
\end{Highlighting}
\end{Shaded}

\section{Mathematical functions}\label{mathematical-functions}

NumPy provides a
\href{https://docs.scipy.org/doc/numpy/reference/ufuncs.html\#available-ufuncs}{vast
choice of mathematical functions} that operate element-wise: among them
\texttt{np.sin}, \texttt{np.cos}, \texttt{np.tan},\\
\texttt{np.log}, \texttt{np.log2}, \texttt{np.log10}, \texttt{np.exp}.

\subsubsection{Exercise 3.}\label{exercise-3.}

Plot the functions \$sin(x)\$ and \$2 \textbackslash{}cdot cos(x) - 1\$
on the interval \${[}-\textbackslash{}pi, +\textbackslash{}pi{]}\$.

\begin{Shaded}
\begin{Highlighting}[]
\CommentTok{# insert code here and evaluate}
\end{Highlighting}
\end{Shaded}

\subsection{Saving and loading}\label{saving-and-loading}

NumPy provides functions
\href{https://docs.scipy.org/doc/numpy/reference/generated/numpy.loadtxt.html\#numpy.loadtxt}{\texttt{array\ =\ np.loadtxt(filename,\ dtype)}}
and
\href{https://docs.scipy.org/doc/numpy/reference/generated/numpy.savetxt.html}{\texttt{np.savetxt(filename,\ array)}}
that should be used to do \emph{efficient} I/O of large arrays.

SciPy additionally provides functions
\href{https://docs.scipy.org/doc/scipy/reference/generated/scipy.io.loadmat.html}{\texttt{scipy.io.loadmat()}}
and
\href{https://docs.scipy.org/doc/scipy/reference/generated/scipy.io.savemat.html}{\texttt{scipy.io.loadmat()}}
to load/save matrices in MATLAB \texttt{.mat} format.

\subsection{Further reading}\label{further-reading}

Nicolas Rougier has written excellent tutorials on the use of Matplotlib
and NumPy:

\begin{itemize}
\tightlist
\item
  \href{https://www.labri.fr/perso/nrougier/teaching/matplotlib/}{Matplotlib
  tutorial}
\item
  \href{http://www.labri.fr/perso/nrougier/teaching/numpy/numpy.html}{NumPy
  tutorial}
\end{itemize}

The Seaborn library comes with a good tutorial written by its author
(note that -since Seaborn is an add-on to Matplotlib- some knowledge of
Matplotlib is assumed):

\begin{itemize}
\tightlist
\item
  \href{http://seaborn.pydata.org/tutorial.html}{Seaborn tutorial}
\end{itemize}
