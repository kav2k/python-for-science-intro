\documentclass[english,serif,mathserif,xcolor=pdftex,dvipsnames,table]{beamer}

\usepackage[T1]{fontenc}
\usepackage[utf8]{inputenc}

\usetheme[informal]{s3it}
\usepackage{s3it}

\title[6. Dictionaries]{%
  A Short and Incomplete Introduction to Python
}
\subtitle{\bfseries Part 7: \texttt{dict}s and other data structures}
\author[R.~Murri]{%
  \textbf{Riccardo Murri} \texttt{<riccardo.murri@uzh.ch>}, \\
  Sergio Maffioletti \texttt{<sergio.maffioletti@uzh.ch>}
  \\
  S3IT: Services and Support for Science IT,
  \\
  University of Zurich
}
\date{January 25--26, 2018}


\begin{document}

% title frame
\maketitle


\part{Dictionaries}

%%% dictionaries

\begin{frame}[fragile]
  \frametitle{Dictionaries}
  The \texttt{dict} type implements a key/value mapping:
\begin{lstlisting}
>>> D = { }
>>> D['a'] = 1
>>> D[2] = 'b'
>>> D
{'a': 1, 2: 'b'}
\end{lstlisting}

\+
  Dictionaries can be created and initialized using the following syntax:
\begin{lstlisting}
>>> D = { 'a':1, 2:'b' }
>>> D['a']
1
\end{lstlisting}
\end{frame}

\begin{frame}[fragile]
  The \texttt{for} statement can be used to loop over keys of a dictionary:
  \+
  \begin{columns}[c]
    \begin{column}{0.5\textwidth}
\begin{lstlisting}
>>> D = { 'a':1, 'b':2 }
>>> for val in ~\HL{D.keys()}~:
...   print(val)
'a'
'b'
\end{lstlisting}
    \end{column}
    \begin{column}{0.4\textwidth}
      \raggedleft
      Loop over dictionary~\emph{keys}.

      \emph{The \texttt{.keys()} part can be omitted, as it's the
        default!}
    \end{column}
  \end{columns}
\end{frame}

\begin{frame}[fragile]
  If you want to loop over dictionary \emph{values}, you have to
  explicitly request it.

  \+
  \begin{columns}[c]
    \begin{column}{0.5\textwidth}
\begin{lstlisting}
>>> D = dict(a=1, b=2)
>>> for val in ~\HL{D.values()}~:
...   print(val)
1
2
\end{lstlisting}
    \end{column}
    \begin{column}{0.4\textwidth}
      \raggedleft
      Loop over dictionary~\emph{values}

      \emph{The \texttt{.values()} cannot be omitted!}
    \end{column}
  \end{columns}
\end{frame}


%%% Mutable vs immutable

\begin{frame}[fragile]
  \frametitle{Mutable vs Immutable}
  Some objects (e.g., \texttt{tuple}, \texttt{int}, \texttt{str})
  are \emph{immutable} and cannot be modified.
\begin{lstlisting}
>>> S = 'UZH'
>>> S[2] = 'G'
Traceback (most recent call last):
  File "<stdin>", line 1, in <module>
TypeError: 'str' object does not support item assignment
\end{lstlisting}


  \+
  \texttt{list}, \texttt{dict}, \texttt{set} and user-defined objects
  are \emph{mutable} and can be modified in-place.
\end{frame}

\begin{frame}[fragile]
  \frametitle{Dictionary, sets and mutable objects}

  Not all objects can be used as dictionary \emph{keys} or items in a
  set:
  \begin{itemize}
    \item
      \textit{Immutable} objects \textbf{can be} used as \texttt{dict} keys or set items.
    \item
      \textit{Mutable} objects  \textbf{cannot be} used as \texttt{dict} keys or set items.
    \end{itemize}

    \+
    {\footnotesize
      (Explanation for the technically savvy: a dictionary is
      essentially a \href{http://en.wikipedia.org/wiki/Hash_table}{Hash
        Table}, therefore keys of a dictionary must be \textit{hashable}
      objects.  If objects were allowed to mutate, their hash value
      would change too and we would lose the mapping.)}
\end{frame}


\begin{frame}[fragile]
  \frametitle{The `{\ttfamily\bfseries in}' operator (1)}

  Use the \lstinline|in| operator to test for presence of an item in a
  collection.

  \begin{describe}{\lstinline|x in S|}
    Evaluates to \texttt{True} if \lstinline|x| is equal to a \emph{value}
    contained in the \lstinline|S| sequence (list, tuple, set).
  \end{describe}

  \begin{describe}{\lstinline|S in T|}
    Evaluates to \texttt{True} if \lstinline|S| is a substring of
    string \lstinline|T|.
  \end{describe}

\end{frame}


\begin{frame}[fragile]
  \frametitle{The `{\ttfamily\bfseries in}' operator (2)}

  Use the \lstinline|in| operator to test for presence of an item in a
  collection.

  \begin{describe}{\lstinline|x in D| \\ \lstinline|x in D.keys()|}
    Evaluates to \texttt{True} if \lstinline|x| is equal to a \emph{key}
    in the \lstinline|D| dictionary.
  \end{describe}

  \begin{describe}{\lstinline|x in D.values()|}
    Evaluates to \texttt{True} if \lstinline|x| is equal to a \emph{value}
    in the \lstinline|D| dictionary.
  \end{describe}

\end{frame}


\begin{frame}[fragile]
\begin{exercise*}[7.A]
    Write a function \lstinline|wordcount(filename)| that reads a text
    file and returns a dictionary, mapping words into occurrences
    (disregarding case) of that word in the text.

    \+ For example, using the
    \href{https://raw.github.com/gc3-uzh-ch/python-course/master/lorem_ipsum.txt}{lorem\_ipsum.txt}
    file:
    \begin{lstlisting}
>>> wordcount('lipsum.txt')
{'and': 3, 'model': 1, 'more-or-less': 1,
 'letters': 1, ~{\em [...]}~
    \end{lstlisting}

    \+ For the purposes of this
    exercise, a ``word'' is defined as a sequence of letters and the
    character ``-'', i.e., ``e-mail'' and ``more-or-less'' should both
    be counted as a single word.
  \end{exercise*}
\end{frame}


\part{Appendix}


\subsection{Making copies}

\begin{frame}[fragile]
  \frametitle{How to copy an object?}
  \begin{lstlisting}
>>> import copy
>>> a = [1, 2]
>>> b = copy.copy(a)
>>> b.remove(1)
>>> print(b)
[2]
>>> print(a)
[1, 2]
  \end{lstlisting}
\end{frame}


\begin{frame}[fragile]
  \frametitle{How to copy an object? (2)}
Note that \texttt{copy.copy} makes a \emph{shallow} copy:
  \begin{lstlisting}
>>> D = { 'a':[1,2], 'b':3 }
>>> print(D['a'])
[1, 2]
>>> E = copy.copy(D)
>>> print(E)
{ 'a':[1, 2], 'b':3 }
>>> E['a'].remove(1)
>>> print(D['a'])
[2]
  \end{lstlisting}
\end{frame}


\begin{frame}[fragile]
  \frametitle{How to copy an object? (3)}
To make a copy of nested data structures, you need \texttt{copy.deepcopy}:
  \begin{lstlisting}
>>> D = { 'a':[1,2], 'b':3 }
>>> print(D['a'])
[1, 2]
>>> E = copy.deepcopy(D)
>>> print(E)
{ 'a':[1, 2], 'b':3 }
>>> E['a'].remove(1)
>>> print(D['a'])
[1, 2]
>>> print(E['a'])
[2]
  \end{lstlisting}
\end{frame}


\end{document}

%%% Local Variables:
%%% mode: latex
%%% TeX-master: t
%%% End:
