\documentclass[english,serif,mathserif,xcolor=pdftex,dvipsnames,table]{beamer}

\usepackage[T1]{fontenc}
\usepackage[utf8]{inputenc}

\usetheme[informal]{s3it}
\usepackage{s3it}

\title[4. Files and strings]{%
  A Short and Incomplete Introduction to Python
}
\subtitle{\bfseries Part 5: File I/O and string processing}
\author[R.~Murri]{%
  \textbf{Riccardo Murri} \texttt{<riccardo.murri@uzh.ch>}, \\
  Sergio Maffioletti \texttt{<sergio.maffioletti@uzh.ch>}
  \\
  S3IT: Services and Support for Science IT,
  \\
  University of Zurich
}
\date{January 25--26, 2018}


\begin{document}

% title frame
\maketitle


\part{File I/O}

\begin{frame}[fragile]
  \frametitle{File I/O}

  Code for processing a text file usually looks like this:
\begin{lstlisting}
with open(filename, 'r') as stream:
  # prepare for processing
  for line in stream:
    # process each line
\end{lstlisting}
\end{frame}


\begin{frame}[fragile]
  \frametitle{File I/O}

\begin{lstlisting}
with ~\HL{open(filename, 'r')}~ as stream:
  # prepare for processing
  for line in stream:
    # process each line
\end{lstlisting}

  \+ The \lstinline|open(path, mode)| function opens the file located at
  \texttt{path} and returns a ``file object'' that can be used for reading
  and/or writing.

  \+ Mode is one of '\texttt{r}', '\texttt{w}' or '\texttt{a}' for reading,
  writing (truncates on open), appending. You can add a `\texttt{+}' character
  to enable read+write (other effects being the same).
\end{frame}


\begin{frame}[fragile]
  \frametitle{File I/O}

\begin{lstlisting}
~\HL{with}~ open(filename, 'r') ~\HL{as}~ stream:
  # prepare for processing
  for line in stream:
    # process each line
\end{lstlisting}

  \+
  This is equivalent to \lstinline|stream = open(...)| but in addition
  \emph{closes} the file when the code in the \texttt{with}-block is done.

  \+
  There are many more uses of the \texttt{with} statement besides automatically
  closing files, check out \url{https://jeffknupp.com/blog/2016/03/07/python-with-context-managers/}
\end{frame}


\begin{frame}[fragile]
  \frametitle{File I/O}

\begin{lstlisting}
with open(filename, 'r') as stream:
  # prepare for processing
  ~\HL{for line in stream}~:
    # process each line
\end{lstlisting}

  \+ A \texttt{for}-loop can be used to process all lines in a file, as if the
  file were a list.
\end{frame}


\begin{frame}[fragile]
  \frametitle{More on File I/O}

  The \lstinline|.read()| method can be used to read the \emph{whole} contents
  of a file in one go as a single string:
\begin{lstlisting}
>>> s = stream.read()
\end{lstlisting}

  \+
  Method \lstinline|.readlines()| returns a list of all lines in the file:
\begin{lstlisting}
>>> L = stream.readlines()
\end{lstlisting}

  \begin{references}
    \url{http://docs.python.org/library/stdtypes.html#file-objects}
  \end{references}
\end{frame}


\begin{frame}[fragile,label=typeconv]
  \frametitle{Type conversions}
  \begin{description}
  \item[str($x$)] Converts the argument $x$ to a string; for numbers,
    the base 10 representation is used.
  \item[int($x$)] Converts its argument $x$ (a number or a string) to an integer;
    if $x$ is a a floating-point literal, decimal digits are truncated.
  \item[float($x$)] Converts its argument $x$ (a number or a string) to a
    floating-point number.
  \end{description}
\end{frame}


\begin{frame}[fragile]
  \begin{exercise*}[5.A]
    Write a function \lstinline|load_data(filename)| that reads a file
    containing one integer number per line, and return a list of the
    integer values.

    \+
    Test it with the
    \href{https://raw.github.com/gc3-uzh-ch/python-course/master/values.txt}{values.txt}
    file:
\begin{lstlisting}
>>> load_data('values.dat')
[299850, 299740, 299900, 300070, 299930]
\end{lstlisting}
  \end{exercise*}
\end{frame}


\begin{frame}[fragile]
  \frametitle{Operations on strings, I}
  \begin{describe}{%
      \lstinline|s.capitalize()|,
      \lstinline|s.lower()|,
      \lstinline|s.upper()|}
    Return a \emph{copy} of the string capitalized / turned all lowercase /
    turned all uppercase.
  \end{describe}

  \begin{describe}{\lstinline|s.split(t)|}
    Split \texttt{s} at every occurrence of \texttt{t} and return a list
    of parts.  If \texttt{t} is omitted, split on whitespace.
  \end{describe}

  \begin{describe}{\lstinline|s.startswith(t)|,
      \lstinline|s.endswith(t)|}
    Return \texttt{True} if \texttt{t} is the initial/final substring
    of \texttt{s}.
  \end{describe}

  \begin{references}
    \url{http://docs.python.org/library/stdtypes.html#string-methods}
  \end{references}
\end{frame}


\begin{frame}[fragile]
  \frametitle{Operations on strings, II}
  \begin{describe}{\lstinline|s.replace(old, new)|}
    Return a \emph{copy} of string \texttt{s} with all occurrences of
    substring \texttt{old} replaced by \texttt{new}.
  \end{describe}

  \begin{describe}{%
      \lstinline|s.lstrip()|,
      \lstinline|s.rstrip()|,
      \lstinline|s.strip()|}
    Return a \emph{copy} of the string with the leading (resp.\ trailing,
    resp.\ leading \emph{and} trailing) whitespace removed.
  \end{describe}

  \begin{references}
    \url{http://docs.python.org/library/stdtypes.html#string-methods}
  \end{references}
\end{frame}


\begin{frame}[fragile]
  \frametitle{Filesystem operations, I}
  \small
  These functions are available from the \texttt{os} module.

  \begin{describe}{\lstinline|os.getcwd()|, \lstinline|os.chdir(path)|}
    Return the path to the current working directory /
    Change the current working directory to \texttt{path}.
  \end{describe}

  \begin{describe}{\lstinline|os.listdir(dir)|}
    Return list of entries in directory \texttt{dir} (omitting
    `\texttt{.}' and `\texttt{..}')
  \end{describe}

  % \begin{describe}{\lstinline|os.mkdir(path)|}
  %   Create a directory; fails if the directory already exists.
  %   Assumes that all parent directories exist already.
  % \end{describe}

  \begin{describe}{\lstinline|os.makedirs(path)|}
    Create a directory; no-op if the directory already exists.
    Creates all the intermediate-level directories needed to contain
    the leaf.
  \end{describe}

  \begin{describe}{\lstinline|os.rename(old,new)|}
    Rename a file or directory from \texttt{old} to \texttt{new}.
  \end{describe}

  \begin{references}
    \url{http://docs.python.org/library/os.html}
  \end{references}
\end{frame}


\begin{frame}[fragile]
  \frametitle{Filesystem operations, II}
  These functions are available from the \texttt{os.path} module.

  \begin{describe}{\lstinline|os.path.exists(path)|, \lstinline|os.path.isdir(path)|, \lstinline|os.path.isfile(path)|}
    Return \texttt{True} if \texttt{path} exists / is a directory / is
    a regular file.
  \end{describe}

  \begin{describe}{\lstinline|os.path.basename(path)|,
      \lstinline|os.path.dirname(path)|}
    Return the base name (the part after the last `\texttt{/}'
    character) or the directory name (the part before the last
    \texttt{/} character).
  \end{describe}

  \begin{describe}{\lstinline|os.path.abspath(path)|}
    Make \texttt{path} absolute (i.e., start with a \texttt{/}).
  \end{describe}

  \begin{references}
    \url{http://docs.python.org/library/os.path.html}
  \end{references}
\end{frame}


\end{document}

%%% Local Variables:
%%% mode: latex
%%% TeX-master: t
%%% End:
