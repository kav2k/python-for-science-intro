\documentclass[english,serif,mathserif,xcolor=pdftex,dvipsnames,table]{beamer}

\usepackage[T1]{fontenc}
\usepackage[utf8]{inputenc}

\usetheme[informal]{s3it}
\usepackage{s3it}

\title[9. Object-orientation]{%
  A Short and Incomplete Introduction to Python
}
\subtitle{\bfseries Part 9: Object-oriented programming}
\author[S.~Haug]{%
  \textbf{Sigve Haug} \texttt{<sigve.haug@math.unibe.ch>}, \\
  Alexander Kashev \texttt{<alexander.kashev@math.unibe.ch>} \\
  Science IT Support (ScITS), University of Bern \\
  \medskip
  Based on a course by Riccardo Murri / Sergio Maffioletti
  \\
  S3IT: Services and Support for Science IT, UZH
}
\date{August 20--21, 2018}


\begin{document}

% title frame
\maketitle


\part{Objects}

\begin{frame}
  \frametitle{What's an \emph{object}?}
  \textbf{A Python object is a bundle of variables and functions.}

  \+
  What variable names and functions comprise an object is defined
  by the object's \emph{class}.

  \+
  From one class specification, many objects can be
  \emph{instanciated}.  Different instances can assign different
  values to the object variables.

  \+
  Variables and functions in an instance are collectively called
  \emph{instance attributes}; functions are also termed \emph{instance
    methods}.
\end{frame}


\begin{frame}[fragile]
  \frametitle{Example: the \texttt{datetime} object, I}
  \begin{columns}[c]
    \begin{column}{0.5\textwidth}
\begin{lstlisting}
>>> from datetime import date
>>> ~\HL{dt1 = date(2012, 9, 28)}~
>>> dt2 = date(2012, 10, 1)
\end{lstlisting}
    \end{column}
    \begin{column}{0.5\textwidth}
      \raggedleft
      To instanciate an object, call the class name like a
      function.
    \end{column}
  \end{columns}
\end{frame}


\begin{frame}[fragile]
  \frametitle{Example: the \texttt{datetime} object, II}
\begin{lstlisting}
>>> dir(dt1)
['__add__', '__class__', ~\ldots~, 'ctime', 'day',
'fromordinal', 'fromtimestamp', 'isocalendar',
'isoformat', 'isoweekday', 'max', 'min', 'month',
'replace', 'resolution', 'strftime', 'timetuple',
'today', 'toordinal', 'weekday', 'year']
\end{lstlisting}

  \+
  The \texttt{dir} function can list all objects attributes.

  \+
  Note there is no distinction between instance variables and
  methods!
\end{frame}


\begin{frame}[fragile]
  \frametitle{Example: the \texttt{datetime} object, III}
  \begin{columns}[c]
    \begin{column}{0.5\textwidth}
\begin{lstlisting}
>>> dt1.day
28
>>> dt1.month
9
>>> dt1.year
2012
\end{lstlisting}
    \end{column}
    \begin{column}{0.5\textwidth}
      \raggedleft
      Access to object attributes is done by suffixing the
      instance name with the attribute name, separated by a dot
      ``\texttt{.}''.
    \end{column}
  \end{columns}
\end{frame}


\begin{frame}[fragile]
  \frametitle{Example: the \texttt{datetime} object, IV}
  \begin{columns}[c]
    \begin{column}[b]{0.5\textwidth}
\begin{lstlisting}
>>> dt1 = date(2012, 9, 28)
>>> dt2 = date(2012, 10, 1)

>>> dt1.day
~\HL{28}~
>>> dt2.day
~\HL{1}~
\end{lstlisting}
    \end{column}
    \begin{column}[b]{0.5\textwidth}
      \raggedleft
      The same attribute can have different
      values in different instances!
    \end{column}
  \end{columns}
\end{frame}


\begin{frame}[fragile]
  \frametitle{Instance methods}
\begin{lstlisting}[showstringspaces=false]
>>> dt1.isoformat()
'2012-09-28'
\end{lstlisting}

  \+
  Invoke an instance method just like any other function.
\end{frame}


\begin{frame}
  \frametitle{Objects \emph{vs} modules}

  Modules are also namespaces of variables and functions.

  \+
  The dot operator `\texttt{.}' is also used to access variables
  and functions from modules.  The \texttt{dir()} function is also
  used to list variables and functions from modules.

  \+
  But each module has \emph{one and only one} instance in a Python
  program.
\end{frame}


\part{User-defined classes}

\begin{frame}[fragile]
  \frametitle{A \emph{2D vector} in Python}
  \begin{columns}[t]
    \begin{column}{0.5\textwidth}
\begin{lstlisting}
class Vector(object):
  """A 2D Vector."""
  def __init__(self, x, y):
    self.x = x
    self.y = y
  def add(self, other):
    return Vector(self.x+other.x,
                  self.y+other.y)
  def mul(self, scalar):
    return Vector(scalar*self.x, scalar*self.y)
  def show(self):
    return ("<%g,%g>" % (self.x, self.y))
\end{lstlisting}
    \end{column}
    \begin{column}{0.5\textwidth}
      \raggedleft
      This code defines a Python object that implements a 2D vector.
    \end{column}
  \end{columns}

  \+
  {\scriptsize Source code available at:
    \url{https://raw.github.com/gc3-uzh-ch/python-course/master/vector.py}}
\end{frame}


\begin{frame}[fragile]
  \frametitle{What does \texttt{Vector} do?}
  \begin{columns}
    \begin{column}[t]{0.5\linewidth}
  We can create vectors by initializing them with the two coordinates
  $(x,y)$:
\begin{lstlisting}
>>> u = Vector(1,0)
>>> v = Vector(0,1)
\end{lstlisting}
    \end{column}
    \begin{column}[t]{0.5\linewidth}
      The \texttt{show} method shows vector coordinates:
\begin{lstlisting}
>>> u.show()
'<1,0>'
>>> v.show()
'<0,1>'
\end{lstlisting}
    \end{column}
  \end{columns}

  \+
  \begin{columns}
    \begin{column}[t]{0.5\linewidth}
      The \texttt{add} method implements vector addition:
\begin{lstlisting}
>>> w = u.add(v)
>>> w.show()
'<1,1>'
\end{lstlisting}
    \end{column}
    \begin{column}[t]{0.5\linewidth}
      The \texttt{mul} method implements scalar multiplication:
\begin{lstlisting}
>>> v2 = v.mul(2)
>>> v2.show()
'<0,2>'
\end{lstlisting}
    \end{column}
  \end{columns}
\end{frame}


\begin{frame}[fragile]
  \frametitle{User-defined classes, I}
  \begin{columns}[t]
    \begin{column}{0.5\textwidth}
\begin{lstlisting}
~\HL{class}~ Vector(object):
  """A 2D Vector."""
  def __init__(self, x, y):
    self.x = x
    self.y = y
  def add(self, other):
    return Vector(self.x+other.x,
                  self.y+other.y)
  def mul(self, scalar):
    return Vector(scalar*self.x, scalar*self.y)
  def show(self):
    return ("<%g,%g>" % (self.x, self.y))
\end{lstlisting}
    \end{column}
    \begin{column}{0.5\textwidth}
      \raggedleft
      A class definition starts with the keyword \texttt{class}.

      The class definition is indented relative to the \texttt{class}
      statement.
    \end{column}
  \end{columns}
\end{frame}


\begin{frame}[fragile]
  \frametitle{User-defined classes, II}
  \begin{columns}[t]
    \begin{column}{0.5\textwidth}
\begin{lstlisting}
class Vector~\HL{(object)}~:
  """A 2D Vector."""
  def __init__(self, x, y):
    self.x = x
    self.y = y
  def add(self, other):
    return Vector(self.x+other.x,
                  self.y+other.y)
  def mul(self, scalar):
    return Vector(scalar*self.x, scalar*self.y)
  def show(self):
    return ("<%g,%g>" % (self.x, self.y))
\end{lstlisting}
    \end{column}
    \begin{column}{0.5\textwidth}
      \raggedleft
      This identifies user-defined~classes.

      \+
      (Do not leave it out or you'll get an ``old-style'' class, which
      is deprecated behavior.)
    \end{column}
  \end{columns}
\end{frame}


\begin{frame}[fragile]
  \frametitle{User-defined classes, II}
  \begin{columns}[t]
    \begin{column}{0.5\textwidth}
\begin{lstlisting}
class Vector(object):
  ~\HL{"""A 2D Vector."""}~
  def __init__(self, x, y):
    self.x = x
    self.y = y
  def add(self, other):
    return Vector(self.x+other.x,
                  self.y+other.y)
  def mul(self, scalar):
    return Vector(scalar*self.x, scalar*self.y)
  def show(self):
    return ("<%g,%g>" % (self.x, self.y))
\end{lstlisting}
    \end{column}
    \begin{column}{0.5\textwidth}
      \raggedleft
      Classes can have docstrings.

      The content of a class docstring will be shown as help text for
      that class.
    \end{column}
  \end{columns}
\end{frame}


\begin{frame}[fragile]
  \frametitle{User-defined classes, IV}
  \begin{columns}[t]
    \begin{column}{0.5\textwidth}
\begin{lstlisting}
class Vector(object):
  """A 2D Vector."""
  ~\HL{\textbf{def} \_\_init\_\_(\textbf{self}, x, y):}~
    self.x = x
    self.y = y
  def add(self, other):
    return Vector(self.x+other.x,
                  self.y+other.y)
  def mul(self, scalar):
    return Vector(scalar*self.x, scalar*self.y)
  def show(self):
    return ("<%g,%g>" % (self.x, self.y))
\end{lstlisting}
    \end{column}
    \begin{column}{0.5\textwidth}
      \raggedleft
      The {\bf def} keyword introduces a method~definition.

      \+
      Every method \emph{must} have at~least one argument,
      named~{\bf self}.
    \end{column}
  \end{columns}
\end{frame}


\begin{frame}[fragile]
  \frametitle{The \texttt{self} argument}

  \textbf{Every method of a Python object always has \texttt{self}
    as first argument.}

  \+
  However, you do not specify it when calling a method: it's
  automatically inserted by Python:
\begin{lstlisting}
>>> class ShowSelf(object):
...   def show(self):
...     print(self)
...
>>> x = ShowSelf() # construct instance
>>> x.show() # `self' automatically inserted!
<__main__.ShowSelf object at 0x299e150>
\end{lstlisting}

  \+
  The \texttt{self} name is a reference to the object instance
  itself.  You \emph{need to} use \texttt{self} when accessing methods
  or attributes of this instance.
\end{frame}


% \begin{frame}
%   \frametitle{No access control}
%   There are no ``public''/``private''/etc. qualifiers for object
%   attributes.

%   \+
%   \textbf{\emph{Any} code can create/read/overwrite/delete \emph{any} attribute on
%     \emph{any} object.}

%   \+
%   There are \emph{conventions}, though:
%   \begin{itemize}
%   \item ``protected'' attributes: \texttt{\_name}
%   \item ``private'' attributes: \texttt{\_\_name}
%   \end{itemize}
%   (But again, note that this is not \emph{enforced} by the system in
%   any way.)

% \end{frame}


\begin{frame}[fragile]
  \frametitle{Name resolution rules, I}
  \small

  Within a function body, names are resolved according to \href{http://stackoverflow.com/questions/291978/short-description-of-python-scoping-rules/292502#292502}{the LEGB rule}:
  \begin{description}
  \item[L] Local scope: any names defined in the current function;
  \item[E] Enclosing function scope: names defined in enclosing
    functions (outermost last);
  \item[G] global scope: names defined in the toplevel of the enclosing module;
  \item[B] Built-in names (i.e., Python's \texttt{\_\_builtins\_\_} module).
  \end{description}

  \+
  \textbf{Any name that is not in one of the above scopes \emph{must}
    be qualified.}

  \+
  So you have to write \texttt{self.x} to reference an attribute in
  this instance, \texttt{datetime.date} to mean a class defined in module
  \texttt{date}, etc.

  % \begin{references}
  %   \url{http://stackoverflow.com/questions/291978/short-description-of-python-scoping-rules/292502#292502}
  % \end{references}
\end{frame}


\begin{frame}[fragile]
  \frametitle{Name resolution rules, II}
  \begin{columns}
    \begin{column}[t]{0.6\linewidth}
\begin{lstlisting}
import datetime as dt

def today():
  ~\HL{td}~ = dt.date.today()
  return "today is " + ~\HL{td}~.isoformat()

def hey(~\HL{name}~):
  print("Hey " + ~\HL{name}~ + "; " + today())

hey("you")
\end{lstlisting}
    \end{column}
    \begin{column}[t]{0.4\linewidth}
      \raggedleft
      Unqualified name within a function: resolves to a local variable.
    \end{column}
  \end{columns}
\end{frame}


\begin{frame}[fragile]
  \frametitle{Name resolution rules, III}
  \begin{columns}
    \begin{column}[t]{0.6\linewidth}
\begin{lstlisting}
import datetime as dt

def today():
  td = dt.date.today()
  return "today is " + td.isoformat()

def hey(name):
  print("Hey " + name + "; " + ~\HL{today}~())

hey("you")
\end{lstlisting}
    \end{column}
    \begin{column}[t]{0.4\linewidth}
      \raggedleft
      Unqualified name: since there is no local variable by that name,
      it resolves to a module-level binding, i.e., to the
      \texttt{today} function defined above.
    \end{column}
  \end{columns}
\end{frame}


\begin{frame}[fragile]
  \frametitle{Name resolution rules, IV}
  \begin{columns}
    \begin{column}[t]{0.6\linewidth}
\begin{lstlisting}
import datetime as ~\HL[Red!25]{dt}~

def today():
  td = ~\HL{dt}~.date.today()
  return "today is " + td.isoformat()

def hey(name):
  print("Hey " + name + "; " + today())

hey("you")
\end{lstlisting}
    \end{column}
    \begin{column}[t]{0.4\linewidth}
      \raggedleft
      Unqualified name: resolves to the \texttt{dt} name created at global scope by the \texttt{import} statement.
    \end{column}
  \end{columns}
\end{frame}


\begin{frame}[fragile]
  \frametitle{Name resolution rules, V}
  \begin{columns}
    \begin{column}[t]{0.6\linewidth}
\begin{lstlisting}
import datetime as dt

def today():
  td = ~\HL{dt.date}~.today()
  return "today is " + td.isoformat()

def hey(name):
  print("Hey " + name + "; " + today())

hey("you")
\end{lstlisting}
    \end{column}
    \begin{column}[t]{0.4\linewidth}
      \raggedleft
      Qualified name: instructs Python to search the
      \texttt{date} attribute within the \texttt{dt} module.
    \end{column}
  \end{columns}
\end{frame}


\begin{frame}[fragile]
  \frametitle{Name resolution rules, VI}
  \begin{columns}
    \begin{column}[t]{0.6\linewidth}
\begin{lstlisting}
import datetime as dt

def today():
  td = dt.date.today()
  return "today is " + ~\HL{td.isoformat}~()

def hey(name):
  print("Hey " + name + "; " + today())

hey("you")
\end{lstlisting}
    \end{column}
    \begin{column}[t]{0.4\linewidth}
      \raggedleft
      Qualified name: Python searches the \texttt{isoformat} attribute
      within the \texttt{td} object instance.
    \end{column}
  \end{columns}
\end{frame}


\begin{frame}[fragile]
  \frametitle{Name resolution rules, VI}
  \begin{columns}
    \begin{column}[t]{0.6\linewidth}
\begin{lstlisting}
class Vector(object):
  def __init__(self, ~\HL[Red!25]{x}~, ~\HL[Red!25]{y}~):
    self.x = ~\HL{x}~
    self.y = ~\HL{y}~
  # ...
\end{lstlisting}
    \end{column}
    \begin{column}[t]{0.4\linewidth}
      \raggedleft
      Unqualified name: resolves to a local variable in
      scope of function \lstinline|__init__|.
    \end{column}
  \end{columns}
\end{frame}


\begin{frame}[fragile]
  \frametitle{Name resolution rules, VII}
  \begin{columns}
    \begin{column}[t]{0.6\linewidth}
\begin{lstlisting}
class Vector(object):
  def __init__(self, x, y):
    ~\HL{self.x}~ = x
    ~\HL{self.y}~ = y
  # ...
\end{lstlisting}
    \end{column}
    \begin{column}[t]{0.4\linewidth}
      \raggedleft
      Qualified names: resolve to attributes in object \lstinline|self|.

      \+ (Actually, \lstinline|self.x = ...| \emph{creates} the
      attribute \lstinline|x| on \lstinline|self| if it does not exist
      yet.)
    \end{column}
  \end{columns}

\end{frame}


\begin{frame}[fragile]
  \frametitle{Object initialization}
  \begin{columns}[t]
    \begin{column}{0.5\textwidth}
\begin{lstlisting}
class Vector(object):
  """A 2D Vector."""
  ~\HL{\textbf{def} \_\_init\_\_(\textbf{self}, x, y):}~
    self.x = x
    self.y = y
  def add(self, other):
    return Vector(self.x+other.x, self.y+other.y)
  def mul(self, scalar):
    return Vector(scalar*self.x, scalar*self.y)
  def show(self):
    return ("<%g,%g>" % (self.x, self.y))
\end{lstlisting}
    \end{column}
    \begin{column}{0.5\textwidth}
      \raggedleft
      The \lstinline|__init__| method has a special
      meaning: it is called when an instance is created.
    \end{column}
  \end{columns}
\end{frame}



\begin{frame}[fragile]
  \frametitle{Constructors}

  The \lstinline|__init__| method is the object constructor.
  It should \emph{never} return any value.

  \+
  You never call \lstinline|__init__| directly, it is invoked by
  Python when a new object is created from the class:
\begin{lstlisting}
# calls Vector.\_\_init\_\_
v = Vector(0,1)
\end{lstlisting}

  \+
  The arguments to \lstinline|__init__| are the arguments you
  should supply when creating a class instance.

  \+
  (Again, minus the \texttt{self} part which is automatically
  inserted by Python.)
\end{frame}


% \begin{frame}
%   \frametitle{No overloading}

%   \textbf{Python does not allow overloading of functions.}

%   \+
%   Any function.

%   \+
%   Hence, no overloading of constructors.

%   \+
%   So: \textbf{a class has one and only one constructor.}
% \end{frame}


\begin{frame}[fragile]
  \begin{exercise*}[9.A]
    Add a new method \texttt{norm} the \texttt{Vector} class: if
    \texttt{v} is an instance of class \texttt{Vector}, then calling
    \texttt{v.norm()} returns the norm $\sqrt{v_x^2 + v_y^2}$ of
    the associated vector.
    (You will need the
    \href{http://docs.python.org/2/library/math.html}{math} standard
    module for computing square roots.)
  \end{exercise*}

  \+
  \begin{exercise*}[9.B]
    Add a new method \texttt{unit} to the \texttt{Vector} class: if
    \texttt{v} is an instance of class \texttt{Vector}, then calling
    \texttt{v.unit()} returns the vector \texttt{u} having the
    same direction as \texttt{v} but norm $1$.
  \end{exercise*}
\end{frame}


\part{Appendix}

\begin{frame}[fragile]
  \frametitle{Everything is an object!}

  The \texttt{dir} built-in function is used to list the attributes of an object.

  \begin{python}
>>> dir("hello!")
  \end{python}
\end{frame}


\begin{frame}[fragile]
  \frametitle{Everything is an object!}

  The \texttt{dir} built-in function is used to list the attributes of an object.

  \begin{python}
>>> ~\HL{dir("hello!")}~
['__add__', '__class__', '__contains__',
 '__delattr__', '__doc__', '__eq__',
 ~\ldots~
'strip', 'swapcase', 'title',
'translate', 'upper', 'zfill']
\end{python}

\+\ldots a string is an object!
\end{frame}


\begin{frame}[fragile]
  \frametitle{Everything is an object!}

  \begin{python}
>>> ~\HL{dir([1,2,3])}~
['__add__', '__class__', '__contains__',
~\ldots~
'append', 'count', 'extend',
'index', 'insert', 'pop',
'remove', 'reverse', 'sort']
\end{python}

\+\ldots a \texttt{list} is an object!
\end{frame}


\begin{frame}[fragile]
  \frametitle{Everything is an object!}
  Indeed, you can do:

  \+
  \begin{python}
>>> "hello world!".split()
['hello', 'world!']
\end{python}

\+
\begin{python}
>>> [1,1,2,3,5].count(1)
2
\end{python}
\end{frame}


\begin{frame}[fragile]
  \frametitle{Everything is an object!}

\begin{python}
>>> ~\HL{dir(1)}~
\end{python}

\end{frame}


\begin{frame}[fragile]
  \frametitle{Everything is an object!}

  \begin{python}
>>> ~\HL{dir(1)}~
['__abs__', '__add__', '__and__',
~\ldots~
'conjugate', 'denominator',
'imag', 'numerator', 'real']
\end{python}

\+\ldots an \texttt{int} is an object!

\+
\begin{python}
>>> (1).numerator
2
>>> (1).denominator
1
\end{python}
\end{frame}


\end{document}

%%% Local Variables:
%%% mode: latex
%%% TeX-master: t
%%% End:
